\chapter{Introduction}
One of the clearest functionalities of mobile robots is building maps of its surroundings and navigating through them; this is often referred to as SLAM, Simultaneous Localization and Mapping. There exists, however, certain difficulties when implementing SLAM, mainly, for a robot to localize (know its true position), a very precise map has to be built before; but, for a very precise map to be built, a robot must know exactly where it is. Friction, control loss or small obstacles are often the cause of a bad odometry which leads to a poor estimate of the true position. The Rao-Blackwellized Particle filter (as shown in ~\cite{Doucet} and ~\cite{Murphy}) solves this issue by generating estimates of the possible position (particles) and giving them a weight. Those particles that have a higher weight, because their estimate matches better the reality, will survive, and the rest will die out in the next generation. To keep a continuous amount of particles and not let all of them die out, resampling stages are carried out where those surviving particles reproduce and keep the particle count constant. It is evident that the more particles used, the more possible descriptions of the reality one has and the higher probability of having at least one particle which is almost perfect. This also induces a higher computational necessity and stops the solution from becoming a real-time application.\\
The real issue is, therefore, weighing these particles to know how close they are to the reality. This is known as the \textit{proposal distribution} $\pi$, and in the latter iterations of the RBPF-SLAM solution, optimal techniques have been developed to obtain a more accurate representation, whilst keeping the particle count to a minimum.\\
In summary, what the RBPF-SLAM algorithm does is given an initial estimate of the true pose, it will use the high precision of a laser scan and the latest map generation to converge the estimate to the true position. This leads to a more accurate map representation and a considerable reduction in errors and number of particles.