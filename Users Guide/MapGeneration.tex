\section{MapGeneration}
\subsection{Introduction}
In a real-world application of the RBPF-SLAM, the real odometry and laser scans of a robot would be used. For a simulation, however, these values need to be generated by the user. To do so, a simulation map will be created, and a simulated robot will move through it following a determined path. Laser scan data will be derived mathematically and will be fed, together with a error-incorporated odometry, into the RBPF-SLAM method, that will generate different particles and maps to better adjust to the reality of the simulation.

\subsection{Creating the map}
The construction of the map is done through individual walls. A matrix \textit{map} (with dimensions 4xn) will represent these walls. The first row corresponds to the x-coordinate of the initial point of the wall; and the second row will be the y-coordinate. Similarly, the third and fourth row are the x and y coordinates of the final wall point, respectively. 

\begin{equation}
	map(:,n) = [x_{initial}; y_{initial}; x_{final}; y_{final}];
\end{equation}

Therefore the user can create whichever map he/she feels fit, as complex or simple as wanted. the result can be visualized uncommenting the next few lines of code.

\subsection{Path generation}

Once the map is complete, the path the robot will follow has to be detailed. This is done in the \textit{via} vector which incorporates the coordinates of points in the path the robot has to travel through. The \textit{MatLab} function \textit{mstraj} by Peter Corke uses the defined \textit{via}, maximum axis-speed limits and a timestep to determine the trajectory (\textit{path}) of the robot.

\subsection{Control parameters}
\label{Errors}
This is precisely where the control aspect of robotics comes into place. A controller has to be developed to make the robot follow the path as desired. in other words, you will pre-program the robot to follow a certain path instead of giving it movement commands at every time step. The parameters that have to be tuned are $d$, $K_p$ and $K_h$. The error signals in the feedback control loop are calculated as follows:

The absolute distance from the desired position (in \textit{path}) and the latest position $x_{t-1}(x,y,\theta)$ given by the latest movement control command $u_{t-1}$: 
\begin{equation}
	e = \sqrt{((path_{1, t}-x_{1, t-1})^2 + (path_{2, t}-x_{2, t-1})^2)} - d;
\end{equation}
The angular error between these two positions:
\begin{equation}
	th = atan2((path_{2, t}-x_{2, t-1}),(path_{1, t}-x_{1, t-1}));
\end{equation}

The new movement control commands will be:
\begin{equation}
	u_{1, t} = K_p \cdot e;
\end{equation}
\begin{equation}
	u_{2, t} = K_h \cdot (angdiff(th,x_{3, t-1}));
\end{equation}

The resulting path after tunning the controller can be visualized commenting out the RBPF line (\textbf{[pf.xh(:,k),pf] = RBPF(x(:,k),y,LRS,u(:,k-1),a,pf, Nsamples,'multinomial\_resampling')}) and uncommenting the true position plotting section that follows it.


